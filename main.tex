\documentclass{article}
\usepackage[utf8]{inputenc}
\usepackage{longtable}
\usepackage[hidelinks]{hyperref}


\title{Should You Take Adderall In College?}
% find \class tag for: Adderall In College - Research Paper (ENGR231 F22 6261 & ENGL&235 6264 - Technical Writing)
\author{ozymacode}
\date{November 2022}

\begin{document}

    \maketitle
    \pagebreak
    \tableofcontents
    \pagebreak
    % table of contents
    % list of abreviations or symbols
    
    \section{Abstract}
        The purpose of this article is to determine if college students should or shouldn't look to Adderall to help them through school. The scope of this article is just the risks, benefits, and appropriate use of Adderall. This includes health impacts, legal risks, the mental effects of the drug, how to minimize the risks and maximize the benefits. Most of the information will be surface level, There will not be any deep dives into data. 
        
        The conclusion reached is that Adderall can be considered for those who struggle with focus and might have ADHD or narcolepsy. However, Adderall is a extremely addictive substance and is dangerous to take outside of how it is prescribed. Inappropriate usage of Adderall will put you at risk for dependency, overdose, heart failure, respiratory failure, depression, and other negative health impacts. The prescribed dosage for Adderall is started at 20 mg for adults over 18. It can be incremented weekly until effect is noticed but the dose should not exceed 40 mg. Regular Adderall can be taken once or twice a day, and Adderall XR once per day.
    
    \section{List of Tables}
        \begin{longtable}[c]{| c | c |}
            \hline
            \multicolumn{2}{| c |}{Effects of Adderall}\\
            \hline
            \endfirsthead
            \hline
            Negative & Positive\\
            \hline
            Addiction \cite{tardner22} & Reduced Hyperactivity \cite{lakhan12} \\
            Depression \cite{tardner22} & Increased Energy \cite{tardner22} \\
            Heart Problems \cite{lakhan12} & Increased Positive Emotion \cite{weyandt18}\\
            Anxiety \cite{tardner22} &\\
            Adderall Neurotoxicity \cite{nguyen22} &\\
            Psychosis \cite{nguyen22} &\\
            Demotivation \cite{nguyen22} &\\
            Irritability \cite{nguyen22} &\\
            Headaches \cite{nguyen22} &\\
            Insomnia \cite{nguyen22} &\\
            \hline
            \caption{There are many serious negative side effects of Adderall, just like any other stimulant. Addiction is among the most easily overlooked and can often onset in just two weeks. The development of physical dependence is also a primary reason for Adderall being classified as a controlled substance. It should be noted however that most of the negative effects occur due to misuse of the drug and don’t typically appear when taken as prescribed \label{long}}\\
        \end{longtable}
        
    \section{Body}
        
        % A discussion of the topic’s background in a way relevant to the scope of your research and article’s focus

        % A discussion of the methodologies people use/d to begin to measure the topic’s issue or to discover the topic’s existence/feasibility 

        % A definition/discussion of the issue(s) that your recommendations will seek to remedy 
        \subsection{Introduction}
        % Talk about what adderall is and its use to treat ADHD and narcosis
            Unlike in Bolivia, where coca leaves are often chewed socially and publicly, stimulants are largely illegal or highly controlled in the USA. However, Adderall is uniquely un-stigmatized compared to its brother and sister stimulants. Adderall usage has been on the rise for many years now and it’s hardly thought to be as dangerous as cocaine or meth. It’s been found that “up to twenty percent of college students had used Adderall or Ritalin”\cite{rolland16}. It's FDA approved to treat Attention-deficit/hyperactivity disorder (ADHD) and narcolepsy. 
            
            We have all heard the horror stories about other stimulants, so there’s certainly some precautions to be taken with Adderall. With this in mind, it is worth taking a look at what the exact effects of the drug are. Adderall has been seen as the go to ‘study drug’ to help students focus, sparking concerns. This report will be examining the risks, benefits, and appropriate use of Adderall. 
        \subsection{Benefits}
            \subsubsection{Focus}
                It's been found that stimulants like Adderall reduce hyperactivity, impulsivity, and in short “help people with ADHD feel more focused.”\cite{lakhan12}. 
                According to Elizabeth Broadbent, Adderall gave her a whole new life. She had been suffering narcolepsy and ADHD. Before her prescription, she said how procrastination was ruining her life. "You have something to do, you don’t want to do it, and you can’t bring yourself to do it". Elizabeth would often snap at her kids, be exhausted to deal with her friends, and avoid social occasions. But after she started taking Adderall everything changed. "I got stuff done, the way neurotypical people do."\cite{broadbent21}. She became far more relaxed with her kids behavior, wouldn't dread doing the laundry or picking up the phone to talk with someone. 
                If focus is something that you really struggle with, then Adderall might be able help you the same way it helped her.
            \subsubsection{Positive Emotion}
                According to Lisa L. Weyandt at the George and Anne Ryan Institute for Neuroscience\cite{weyandt18}, two major effects of Adderall are positive emotion and subjective drug experiences. Does this mean it helps with depression? Unfortunately, the positive states of activated emotion are generally short lived and Adderall is not a FDA approved treatment for depression. There is some evidence that Adderall can treat people without ADHD. However, it has also been found that depression can caused be by Adderall in some cases\cite{tardner22}. Therefore, overusing Adderall to avoid negative states of emotion is considered drug abuse and is ill advised. But if you only seeking to use Adderall for subjective drug experiences, then take great care to avoid dependence, which will be explained more later.
            \subsubsection{Improved Mental Abilities? Maybe Not...}
                 Unfortunately, Adderall does not make you smarter. According to Lakhan and Kirchgessner at the Global Neuroscience Initiative Foundation in Los Angeles\cite{lakhan12}, Adderall helps subjects retain memory of without much connection to the context or meaning, but it has not been found to help subjects actually learn material more then patients without Adderall.
                 
        \subsection{Risks}
            \subsubsection{Dependency}
                \paragraph{What is Dependency?} To start with, we need to take a look at dopamine. Dopamine is a type of neurotransmitter which is released between brain cells. It is commonly associated with the feeling of satisfaction from accomplishing something, eating food, and having sex. When a neuron releases dopamine into the between the gap connecting it to its neighboring neuron, some of the dopamine will be passed through, and some will be taken back.
                % clarify sentence above
                Many drugs such as heroin, methamphetamine, and cocaine prevent the dopamine from being taken back into the neuron, causing a buildup of dopamine in between. This causes positive emotion. However, repeated use causes you to build a tolerance for the dopamine. 
                % clarify tolerance
                Eventually, the drug becomes necessary for maintaining the former baseline for positive emotion. Then when the user stops using the drug, they drop into a period of low dopamine, while they still have a high tolerance. This is called withdrawal. Withdrawal can cause 
                % add other effects
                , in some extreme cases, even cause death. If sudden elimination of the drug will cause a withdrawal, then that user has what is called a physical dependence on the drug.
                
                It should be kept in mind that Adderall is a schedule II narcotic. According the DEA, "Substances in this schedule have a high potential for abuse which may lead to severe psychological or physical dependence."\cite{DOJ}. If you decide to try Adderall, be aware that dependency is likely to appear if you've been on it for over two weeks.\cite{tardner22} 
                % exmplain how common it is, glide into example
                For example, one person who suffered this was Taylor Evans, who did not have ADHD, but she went to a psychiatrist and was proscribed to take Adderall anyway. She was doing fine in school before she started, but after taking Adderall she compared it to "being superwoman"\cite{marshall15}. She would work better and more than she did before. As time went on, her priorities began to slip. She started to be less concerned with school or work and more on getting pills. She gradually stopped attending class, and began getting panic attacks. Her work performance started to drop, she stopped see her friends and family. "You can’t even sit down to complete a task, so in reality, it’s even harder to get stuff done" she said. When she finally went for help, it turned out that she was taking "more than 10 times the highest prescribed dosage of amphetamines per day"\cite{marshall15}. She ended up spending two and a half years battling severe addiction. 
                
                % according to 

            
            \subsubsection{Negative Health Impacts}
                \paragraph{Overuse of Adderall} and other stimulants can cause numerous heart and lung dangers. Respiratory distress syndrome (ards) for example, is where fuild builds up in the lungs air sacks and makes it difficult to breath. Another potential risk is cardiogenic pulmonary edema, which is high pressure in the heart and often leads to heart failure. Another risk is myocardial infarction, where a lack of blood flow causes heart damange and failure. Other dangers caused by Adderall overuse are depression and anxiety. Further, psychological problems such as psychosis\cite{lakhan12} have been known to occur as a result of stimulant abuse. Fortunatly, When Adderall is taken at its prescribed dosage, these problems haven't been seen to occur in relation to Adderall\cite{khan21}.The myriad health risks should be kept in mind if you are taking Adderall in an un-prescribed manner.
            \subsubsection{US Law}
                % Adderall usage is considered cheating in certain universities 

                % "The minimum sentence for a violation after two or more prior convictions for a felony drug offense have become final is a not less than 25 years imprisonment and a fine up to \$20 million if an individual and \$75 million if other than an individual."\cite{dea22}

                \paragraph{Adderall is a controlled substance.} Meaning that you can face serious legal ramifications for possessing it without a proscription, beyond the proscribed amount, or after it's expiration date. Penalties can range from misdemeanors to felonies depending on the quantity, intent of possession, and method of acquisition. Typically, misdemeanors result in probation, fines, and short prison sentences while felonies can land you more than 10 years in prison. If caught trafficking a schedule II drug like Adderall, first time offenders can face "between \$1 million and \$5 million, alongside an incarceration period of up to 20 years in prison"\cite{hindin22}. If caught with a record of trafficking, offenders can face "between \$2 and \$10 million, and up to 30 years in prison"\cite{hindin22}. Because of the high legal penalties involved, you should take great care in how you use it. 
                % tuck next paragraph into above
                While inside of your prescription bottle, possession of Adderall is perfectly legel. However, possession of Adderall outside can face a plethora of legal penalties. Depending on how you are caught, the penalties can change. Some of the different conditions are as follows.
                \begin{itemize}
                    \item Possession Outside Your Proscription Bottle: Though not often prosecuted, this is a criminal offense. In some particularly harsh states such as Texas, you can face a felony offense, between 180 days to two years in jail, as well as a fine up to \$10,000 if caught in possession of even less than a single gram of Adderall outside of its container. Penalties very between states though, and there are exceptions.
                    \item Possession of Someone Else’s Adderall Prescription: Illegal federally. In California for example, "the unlawful “simple” possession of Adderall (whether somebody else’s prescription or otherwise) is a misdemeanor, with maximum penalties of one year in county jail and a \$1,000 fine."\cite{hindin22}
                    \item Getting Caught with Adderall at School: Most of the criminal penalties remain the same, however if you have quantities constituting an intent to distribute you will face more severe penalties. In addition, it should be noted that school has the right to suspend, expel or refer you to the authorities at their discretion.
                    \item Being Caught on Adderall: It is also illegal to use Adderall without a prescription. In Colorado, "Adderall is classified as a level 2 misdemeanor, with penalties of up to 12 months in prison and a \$750 fine"\cite{hindin22}. If on probation, and faced with a random drug test, the penalties can be much more severe. Adderall is "detectable for up to 7 days in urine tests, 24 hours in blood tests, and up to 90 days in hair follicle tests."\cite{hindin22}
                \end{itemize}




                
            \subsubsection{International Legalities}
                If you decide to travel, be carefull to research the country you are visiting. The United states may be more strict then bolivia, but countries like Japan are far more so. Carrie Russell, a woman who went to japan to teach, had her Adderall prescription shipped to her from America. Soon after, japanese officers had her arrested for trafficing\cite{ore15}. Even if you possess an american prescription, you can still face jail time for possession of Adderall. 
                
                
        \subsection{Appropriate Use}
            \subsubsection{How To Take Adderall}
                According to Kenneth S. Fill, PharmD, MBA\cite{fill22}, Adderall is proscribed either as regular Adderall or extended-release Adderall XR. Both forms are proscribed at dosages between 5 mg to 30 mg. Dosages may need to be adjusted over time. For regular Adderall, adults are typically proscribed once or twice daily, and just once daily for Adderall XR. Both can be taken with or without food.
                The standard starting dose for regular Adderal on patients 6 or older is 5 mg once or twice daily, increasing by 5 mg every week until noticeable effects are seen. For children under 6 starting doses are 2.5 mg with increments of 2.5 mg weekly until effects are seen.
                For Adderall XR, the starting dose for adults 18 and older is often 20 mg with increments until effects are noticed.
                It is advised to take Adderall at the same time ever day to get the most benefits.
                It is highly ill advised to snort it or inject it. When Adderall is snorted or otherwise abused, "the functions of the central nervous system may be increased to hazardous levels."\cite{scot22}. This can cause an overdose. Which can potentially lead to "coma, brain damage, or even death"\cite{scot22}. It is only safe to use Adderall in its prescribed method.
                
            \subsubsection{Risk Factors For Adderall Use}
                As a stimulant, Adderall will increase blood pressure and heart rate which can be dangerous for people who are at risk for cardiovascular disease. Adderall may also have unpredictable effects if you suffer from mental illness. If you do suffer from mental illness, take care to speak with a professional. If you are pregnant or thinking of becoming pregnant, Adderall might not be the right choice. 
                % look into effects
                
                Lastly, there are other drugs can influence Adderall in strange ways. 
                %for quotes, add: according to 
                There are "more than 180 medications known to interact with Adderall"\cite{fill22}, many of them risky or down right dangerous. It is best to talk with your doctor at regular intervals to ensure your proscription is working as it should
            \subsubsection{How Many People Deal With Addiction To Adderall?}
                % use survey about how many people consider addiction as risk
                According to the National Survey on Drug Use and Health (NSDUH), among people 12 and older, there were 18.4 million people in the USA who suffered from substance use disorder from a illicit drugs in 2020. 3.2 million of them suffered from Central Nervous System Stimulants, which Adderall is classified under\cite{samhsa20}.
                
                % add more to this what does the evidence cited tell you?
        \subsection{Conclusion}
            % A discussion of the findings and conclusions regarding the impact 
            Adderall has helped many people in the past conquer their struggles with procrastination. But addiction is a real danger which is not often thought about.
            It is easy to think of Adderall as just study drug, but it is a section II narcotic for a reason. So keep in mind that it is a double edge sword which can harm you if it is not treated with respect.

            Adderall is a stimulant which, if used inappropriately, has been found to cause dependence, depression and other negative health up to and including sudden death. However it has also been found that Adderall reduces hyperactivity, and impulsivity. And in appropriate dosages, many of the negative effects can be eliminated and it can even reduce depression. Lastly, Adderall should be started at 20 mg for adults over 18, but should not exceed 40 mg. Regular Adderall can be taken once or twice a day, and Adderall XR once per day. Adderall taken other then the proscribed method can cause overdose. Adderall does help people focus but the risks are not negligible and should be taken seriously.


        
   
    \pagebreak
    \section{Appendices}
        I've been interested in Adderall for a while and this research paper was pretty enlightening for me. I didn't realize how serious Adderall addiction was before writing it. I was under the impression that Adderall was a relatively low risk drug but turns out it's entirely easy to slip into abuse and dependence. It's still extremely attractive for a easily distracted college student though, but I look at with more caution then I did before writing this. Another pretty shocking fact is that people in Bolivia really do just chew coco leaves casually. That gave me a bit of culture shock. Anyways, thanks for reading!
        
        PS: I used LaTex on overleaf.com to write this paper. It's super helpful. Consider it if you're writing technical content in the future. Not too steep of a learning curve either.
    \section{Glossary}
        Adderall XR - Adderall Extended Release
    
    \section{References}
        % the issue(s) has/have upon an industry and/or overall society ____ An explicit discussion of the feasible recommendation(s) people must consider as they try to determine ways to remedy the topic’s issue(s) 
        \renewcommand{\section}[2]{}%
        \begin{thebibliography}{}
            \bibitem{joyce07}
                Joyce, Matthew B. “Adderall Produces Increased Striatal Dopamine Release and a Prolonged Time Course Compared to Amphetamine Isomers.” Login, Germany: Springer Nature B.V, 2007, \url{https://sbctc-edcc.primo.exlibrisgroup.com/discovery/fulldisplay?docid=cdi_proquest_miscellaneous_70228008&amp;context=PC&amp;vid=01STATEWA_EDCC%3AEDMONDS&amp;lang=en&amp;search_scope=MyInst_and_CI&amp;adaptor=Primo+Central&amp;tab=Everything&amp;query=any%2Ccontains%2Cadderall+effect&amp;offset=0}
            \bibitem{khan21}
                Khan, Alisha. “Routine Use of Prescription Adderall Leading to Non-Cardiogenic Pulmonary Edema and Respiratory Failure.” Login, Palo Alto (CA): Cureus, 2021, \url{https://sbctc-edcc.primo.exlibrisgroup.com/discovery/fulldisplay?docid=cdi_pubmedcentral_primary_oai_pubmedcentral_nih_gov_8360763&amp;context=PC&amp;vid=01STATEWA_EDCC%3AEDMONDS&amp;lang=en&amp;search_scope=MyInst_and_CI&amp;adaptor=Primo+Central&amp;tab=Everything&amp;query=any%2Ccontains%2Cadderall+effect&amp;offset=0}
            \bibitem{lakhan12}
                Lakhan, S. E. and Kirchgessner, A. (2012, July 23). Prescription stimulants in individuals with and ... - Wiley Online Library. Prescription stimulants in individuals with and without attention deficit hyperactivity disorder: misuse, cognitive impact, and adverse effects. Retrieved October 17, 2022, from \url{https://onlinelibrary.wiley.com/doi/full/10.1002/brb3.78}
            \bibitem{marshall15}
                Marshall, Olivia. “ The Real Effects of Adderall: a Personal Testimony.” Login, Carlsbad: Uloop, Inc, 2015, 
                \url{https://sbctc-edcc.primo.exlibrisgroup.com/discovery/fulldisplay?docid=cdi_proquest_wirefeeds_1715703263&amp;context=PC&amp;vid=01STATEWA_EDCC%3AEDMONDS&amp;lang=en&amp;search_scope=MyInst_and_CI&amp;adaptor=Primo+Central&amp;tab=Everything&amp;query=any%2Ccontains%2Cadderall+effect&amp;offset=10}
            \bibitem{mccabe04}
                McCabe, Sean E, et al. “Non-Medical Use of Prescription Stimulants among US College Students: Prevalence and Correlates from a National Survey.” Https://Owl.purdue.edu/, Oxford, UK: Blackwell Publishing, 31 Aug. 2004, \url{https://sbctc-edcc.primo.exlibrisgroup.com/discovery/fulldisplay?docid=cdi_proquest_miscellaneous_57124875&amp;context=PC&amp;vid=01STATEWA_EDCC%3AEDMONDS&amp;lang=en&amp;search_scope=MyInst_and_CI&amp;adaptor=Primo+Central&amp;tab=Everything&amp;query=any%2Ccontains%2Cadderall+college&amp;offset=0}
            \bibitem{rolland16}
                Rolland, A. D. and Smith, P. J. (2016, November 30). Aided by adderall: Illicit use of ADHD medications by college students. Journal of the National Collegiate Honors Council. Retrieved October 12, 2022, from \url{https://eric.ed.gov/?id=EJ1222138}
            \bibitem{tardner22}
                Tardner, Paul (2022, April 20). Adderall and depression • IJEST. IJEST. Retrieved October 12, 2022, from \url{https://www.ijest.org/nootropics/adderall-and-depression/}
            \bibitem{weyandt18}
                Weyandt, Lisa L. “Neurocognitive, Autonomic, and Mood Effects of Adderall: A Pilot Study of Healthy College Students.” Login, Switzerland: MDPI, 2018, \url{https://sbctc-edcc.primo.exlibrisgroup.com/discovery/fulldisplay?docid=cdi_doaj_primary_oai_doaj_org_article_84c8fa9af2384d6493c94a78d4516fd4&amp;context=PC&amp;vid=01STATEWA_EDCC%3AEDMONDS&amp;lang=en&amp;search_scope=MyInst_and_CI&amp;adaptor=Primo+Central&amp;tab=Everything&amp;query=any%2Ccontains%2Cadderall+college&amp;offset=0}  
            \bibitem{nguyen22}
                Nguyen, Victor. “Adderall: Side Effects, Dosage, with Alcohol, and More.” Medical News Today, MediLexicon International, 4 Oct. 2022, \url{https://www.medicalnewstoday.com/articles/326219#generic}
            \bibitem{broadbent21}
                Broadbent, Elizabeth. “Eureka! Adderall Gave Me a Whole New Life.” ADDitude, ADDitude, 2 Feb. 2021, \url{https://www.additudemag.com/eureka-adhd-meds-gave-me-a-whole-new-life/}
            \bibitem{DOJ}
                Controlled Substance Schedules, US Department of Justice, \url{https://www.deadiversion.usdoj.gov/schedules/#:~:text=Examples%20of%20Schedule%20II%20narcotics,opium%2C%20codeine%2C%20and%20hydrocodone}
            \bibitem{dea22}
                “Federal Trafficking Penalties - Campus Drug Prevention.” Campusdrugprevention.gov, DEA, 3 June 2022, \url{https://www.campusdrugprevention.gov/sites/default/files/2022-07/Federal_Trafficking_Penalties_Chart_6-23-22.pdf}
            \bibitem{hindin22}
                Hindin, Bryan. “What Happens When You Get Caught with Adderall? Legal Consequences and More.” The Recovery Village Drug and Alcohol Rehab, 2 May 2022, \url{https://www.therecoveryvillage.com/adderall-addiction/what-happens-getting-caught/#:~:text=First%2Dtime%20offenders%20in%20violation,to%2020%20years%20in%20prison}
            \bibitem{fill22}
                 Fill, Kenneth S. “Adderall: Uses, Dosage, Interactions \& Safety Information.” Drugwatch.com, 11 Nov. 2022, \url{https://www.drugwatch.com/adderall/#:~:text=Adderall%20comes%20in%20the%20following,t%20approved%20for%20treating%20narcolepsy}
            \bibitem{scot22}
                Thomas, Scot. “What Are the Dangers of Snorting Adderall? – Signs of Overdose.” American Addiction Centers, 8 Sept. 2022, \url{https://americanaddictioncenters.org/adderall/snorting}
            \bibitem{samhsa20}
                “2020 National Survey of Drug Use and Health (NSDUH) Releases.” SAMHSA.gov, \url{https://www.samhsa.gov/data/release/2020-national-survey-drug-use-and-health-nsduh-releases}
            \bibitem{ore15}
                Oregonian/OregonLive, Richard Read | The. “Oregon Woman Jailed in Japan for a Bottle of Adderall.” Oregonlive, 1 Mar. 2015, \url{https://www.oregonlive.com/education/2015/03/a_bottle_of_prescribed_adderal.html} 
        \end{thebibliography}
        
        
\end{document}